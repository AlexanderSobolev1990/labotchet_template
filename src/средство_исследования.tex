\newpage
\section{Средство исследования}\label{section:Средство}

Средством исследования является что-то очень важное и перспективное.

%%%%%%%%%%%%%%%%%%%%%%%%%%%%%%%%%%%%%%%%%%%%%%%%%%%%%%%%%%%%%%%%%%%%%%%%%%%%%%%%%%%%%%%%%%%%%%%%%%%%%%%%%%%%%%%%%%%%%%%%%%%%%%%%%%%%%%%%
\subsection{Методы и принципы, используемые в работе}

\begin{enumerate}
	\item[--] метод 1; 
	\item[--] метод 2; 
	\item[--] метод 3. 
\end{enumerate}
%%%%%%%%%%%%%%%%%%%%%%%%%%%%%%%%%%%%%%%%%%%%%%%%%%%%%%%%%%%%%%%%%%%%%%%%%%%%%%%%%%%%%%%%%%%%%%%%%%%%%%%%%%%%%%%%%%%%%%%%%%%%%%%%%%%%%%%%
\subsection{Изменения и улучшения}

По сравнению с предыдущими версиями внесены следующие значительные изменения и улучшения:
\begin{enumerate}
	\item[--] изменение 1;
	\item[--] изменение 2;
	\item[--] изменение 3.
\end{enumerate}	

\subsection{Программная реализация}
%%%%%%%%%%%%%%%%%%%%%%%%%%%%%%%%%%%%%%%%%%%%%%%%%%%%%%%%%%%%%%%%%%%%%%%%%%%%%%%%%%%%%%%%%%%%%%%%%%%%%%%%%%%%%%%%%%%%%%%%%%%%%%%%%%%%%%%%
\subsubsection{Структурная схема}

Структурная схема состоит из следующих блоков:
\begin{enumerate}
	\item[--] блок 1;
	\item[--] блок 2;
	\item[--] блок 3.
\end{enumerate}

%%%%%%%%%%%%%%%%%%%%%%%%%%%%%%%%%%%%%%%%%%%%%%%%%%%%%%%%%%%%%%%%%%%%%%%%%%%%%%%%%%%%%%%%%%%%%%%%%%%%%%%%%%%%%%%%%%%%%%%%%%%%%%%%%%%%%%%%
\subsubsection{Сборка и установка}

Сборка осуществляется выполнением команды

\noindent\lstinline|mkdir build && cd build && cmake .. && make && cpack|

\noindent из корневой директории проекта, после чего в директории \lstinline|build| появляются собранные пакеты.

Установка пакетов на целевой \acsu{эвм} осуществляется командой

\noindent\lstinline|sudo apt install <название пакета>|







 











